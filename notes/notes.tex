\documentclass{article}

\usepackage{mathtools}
\usepackage{amsthm}

\newcommand{\refl}[1]{\mathsf{refl}_{#1}}
\newcommand{\id}{\mathsf{id}}
\newcommand{\inverse}{\mathsf{inverse}}
\newcommand{\ninverse}[1]{\inverse_{#1}}

\theoremstyle{plain}
\newtheorem{theorem}{Theorem}
\newtheorem{proposition}[theorem]{Proposition}
\newtheorem{lemma}[theorem]{Lemma}
\newtheorem{corollary}[theorem]{Corollary}

\DeclareMathOperator{\adj}{Adj}

\DeclarePairedDelimiter{\p}{(}{)}

\begin{document}

Define \[\adj(f, g) := \prod_{x : A}\prod_{y : B}\p{f(x) = y} \to \p{x = g(y)}.\] Suppose
that \(g\) is a left inverse of \(f\), witnessed by \(G : g \circ f \sim \id_{A}\). We then
construct \(r_{G} : \adj(f, g)\), given by induction by
\[r_{G}(x, f(x), \refl{f(x)}) = H(x).\] This map induces an equivalence between
\(g \circ f \sim \id_{A}\) and \(\adj(f, g)\)

Now, suppose that the type of left inverses of a function \(f\)
\[\sum_{g : B \to A}g \circ f \sim \id_{A}\]
is contractible onto \(\p{h, H}\).
By applying the equivalence between
\(g \circ f \sim \id_{A}\) and \(\adj(f, g)\), we then obtain that the type
\[\sum_{g : B \to A}\adj(f, g)\]
is contractible onto \(\p{h, r_{H}}\).

Consider the identity function \(\id_{A}\). Its type of left inverses is contractible onto
\(\p{\id_{A}, \lambda x. \refl{x}}\).
It then follows that the type
\[\sum_{g : A \to A}\adj(\id_{A}, g)\]
is contractible onto \(\p{\id_{A}, r_{\lambda x. \refl{x}}}\). We now claim that
\(r_{\lambda x. \refl{x}}\) is equal to \(\lambda x y. \id_{x = y}\). By function
extensionality, it suffices to show that \[r_{\lambda x. \refl{x}}(x, y, p) = p\] for all
\(x, y : A\) and \(p : x = y\). By path induction, it then suffices to show that
\[r_{\lambda x. \refl{x}}(x, x, \refl{x}) = \refl{x}\] for all \(x : A\), but this then
follows by definiton of \(r\).
We conclude that \(\sum_{g : A \to A}\adj(\id_{A}, g)\) is contractible onto
\(\p{\id_{A}, \lambda x y. \id_{x = y}}\).

With this, we can now show that \(\ninverse{n + 1}(\id_{A})\) is equal to \(\prod_{x y : A}\ninverse{n}(\id_{x = y})\).
Unfolding \(\ninverse{n + 1}\), we see that
\[\ninverse{n + 1}(\id_{A}) = \sum_{g : A \to A}\sum_{r : \adj(\id_{A}, g)}\prod_{x : A}\prod_{y : B}\ninverse{n}(r(x, y)).\]
The result then follows by reassociating the sigma types and applying the above
contractibility result.
\end{document}

%%% Local Variables:
%%% mode: LaTeX
%%% TeX-master: t
%%% End:
