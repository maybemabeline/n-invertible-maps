\documentclass{article}

\usepackage{mathtools}
\usepackage{amsthm}

\newcommand{\refl}[1]{\mathsf{refl}_{#1}}
\newcommand{\id}{\mathsf{id}}
\newcommand{\inverse}{\mathsf{inverse}}
\newcommand{\ninverse}[1]{\inverse_{#1}}
\newcommand{\tr}{\mathsf{tr}}

\theoremstyle{plain}
\newtheorem{theorem}{Theorem}
\newtheorem{proposition}[theorem]{Proposition}
\newtheorem{lemma}[theorem]{Lemma}
\newtheorem{corollary}[theorem]{Corollary}

\DeclareMathOperator{\adj}{Adj}

\DeclarePairedDelimiter{\p}{(}{)}

\begin{document}

Define \[\adj(f, g) := \prod_{x : A}\prod_{y : B}\p{f(x) = y} \to \p{x = g(y)}.\] Suppose
that \(g\) is a left inverse of \(f\), witnessed by \(G : g \circ f \sim \id_{A}\). We then
construct \(r_{G} : \adj(f, g)\), given by induction by
\[r_{G}(x, f(x), \refl{f(x)}) = H(x).\] This assignment induces an equivalence between
\(g \circ f \sim \id_{A}\) and \(\adj(f, g)\)

Now, suppose that the type of left inverses of a function \(f\)
\[\sum_{g : B \to A}g \circ f \sim \id_{A}\]
is contractible onto \(\p{h, H}\).
By applying the equivalence between
\(g \circ f \sim \id_{A}\) and \(\adj(f, g)\), we then obtain that the type
\[\sum_{g : B \to A}\adj(f, g)\]
is contractible onto \(\p{h, r_{H}}\).

Consider the identity function \(\id_{A}\). Its type of left inverses is contractible onto
\(\p{\id_{A}, \lambda x. \refl{x}}\).
It then follows that the type
\[\sum_{g : A \to A}\adj(\id_{A}, g)\]
is contractible onto \(\p{\id_{A}, r_{\lambda x. \refl{x}}}\). We now claim that
\(r_{\lambda x. \refl{x}}\) is equal to \(\lambda x y. \id_{x = y}\). By function
extensionality, it suffices to show that \[r_{\lambda x. \refl{x}}(x, y, p) = p\] for all
\(x, y : A\) and \(p : x = y\). By path induction, it then suffices to show that
\[r_{\lambda x. \refl{x}}(x, x, \refl{x}) = \refl{x}\] for all \(x : A\), but this then
follows by definiton of \(r\).
We conclude that \(\sum_{g : A \to A}\adj(\id_{A}, g)\) is contractible onto
\(\p{\id_{A}, \lambda x y. \id_{x = y}}\).

With this, we can now show that \(\ninverse{n + 1}(\id_{A})\) is equal to \(\prod_{x, y : A}\ninverse{n}(\id_{x = y})\).
Unfolding \(\ninverse{n + 1}\), we see that
\[\ninverse{n + 1}(\id_{A}) = \sum_{g : A \to A}\sum_{r : \adj(\id_{A}, g)}\prod_{x : A}\prod_{y : B}\ninverse{n}(r(x, y)).\]
The result then follows by reassociating the sigma types and applying the above
contractibility result.
\pagebreak

In the formalisation I actually obtain the contractibility result by a more direct
computational argument, but I thought the above may be be nicer for a paper proof. As in
the formalisation though, we can directly show that the type
\[\sum_{g : A \to A}\adj(\id_{A}, g)\]
is contractible onto \(\p{\id_{A}, \lambda x y. \id_{x = y}}\) in the following way:

We first characterise identifications in the above sigma type. To give an identification
between \(\p{g, r}\) and \(\p{h, s}\), we claim that it is sufficient to give a homotopy
\(H : g \sim h\) and a witness to the commutative triangle
\[r(x, y, p) \cdot H(y) = s(x, y, p)\] for all \(x, y : A\) and \(p : x = y\). The claim
follows by first applying homotopy induction on \(H\) and then noting that the
commutativity data turns into just a homotopy between \(r\) and \(s\), on which we may
again apply homotopy induction.

Now let \(\p{g, r}\) be an arbitrary element of the above sigma type. Then
\[r : \prod_{x, y : A} (x = y) \to (x = g(y)),\] so \(\lambda x. r(x, x, \refl{x})\) gives
a homotopy between \(\id_{A}\) and \(g\). To give an identification between
\(\p{\id_{A}, \lambda x y. \id_{x = y}}\) and \(\p{g, r}\), the characterisation now tells
us that it is sufficient to give an identification of type
\[p \cdot r(y, y, \refl{y}) = r(x, y, p)\]
for all \(x, y : A\) and \(p : x = y\). This is then discharged by path induction on \(p\).

Let me now turn to a potentially interesting thread, related to \(\infty\)-invertibility.
Taking the coinductive definition of \(\infty\)-invertibility, we can unfold
\(\ninverse{\infty}(\id_{A})\) once to obtain
\[\ninverse{\infty}(\id_{A}) = \sum_{g : A \to A}\sum_{r : \adj(\id_{A}, g)}\prod_{x :
    A}\prod_{y : B}\ninverse{\infty}(r(x, y)).\] As in the proof that
\(\ninverse{n + 1}(\id_{A}) = \prod_{x, y : A} \ninverse{n}(\id_{x = y})\), we now obtain
that \(\ninverse{\infty}(\id_{A}) = \prod_{x, y : A} \ninverse{\infty}(\id_{x = y})\).

Now, the start of the proof of proposition 17 from the paper tells us that to show that
\(\ninverse{\infty}(f)\) is a proposition for all functions \(f\), it suffices to show
that \(\ninverse{\infty}(\id_{A})\) is contractible.
To understand \(\infty\)-invertibility, it thus seems crucial to understand the
\(\infty\)-invertible structure of the identity function and by the above, this obeys the
following curious recursive equation:

Define \(F(A) = \ninverse{\infty}(\id_{A})\). Then \(F(A) = \prod_{x, y : A}F(x = y)\).
Is there anything we can extract from this fact by itself?
\end{document}

%%% Local Variables:
%%% mode: LaTeX
%%% TeX-master: t
%%% End:
