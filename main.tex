\documentclass{article}

\usepackage{mathtools}
\usepackage{amsfonts}
\usepackage{amsthm}

\theoremstyle{plain}
\newtheorem{theorem}{Theorem}
\newtheorem{proposition}[theorem]{Proposition}
\newtheorem{lemma}[theorem]{Lemma}
\newtheorem{corollary}[theorem]{Corollary}

\theoremstyle{definition}
\newtheorem{definition}[theorem]{Definition}

\newcommand{\id}{\mathrm{id}}
\newcommand{\Id}{\mathrm{Id}}
\newcommand{\isequiv}{\mathrm{equiv}}
\newcommand{\pr}{\mathrm{pr}}
\newcommand{\refl}{\mathrm{refl}}
\newcommand{\sphere}{S^{2}}
\newcommand{\U}{\mathcal{U}}
\newcommand{\N}{\mathbb{N}}
\newcommand{\lspace}[1]{\Omega^{#1}}

\DeclareMathOperator{\inv}{Inv}
\DeclareMathOperator{\inverse}{inverse}

\newcommand{\ninverse}[1]{\inverse_{#1}}

\begin{document}


\section{Basic construction}

\begin{proposition}
  \label{universal-property-sphere}
  For all types \(A\), the type \(\sphere \to A\) is equal to
  \[\sum_{x : A} \refl_{x} = \refl_{x}.\]
\end{proposition}
\begin{proposition}
  \label{family-over-subtype}
  Suppose that \(A\) is a type, \(B\) a family of types over \(A\) and \(P\) a family of mere
  propositions over \(A\), such that we have a family of functions \(B(x) \to P(x)\) for all
  \(x : A\). We then have an equality of types
  \[\sum_{x : A}B(x) = \sum \Bigl(t : \sum_{x : A}P(x)\Bigr) B(\pr_{1}t).\]
\end{proposition}

A function \(f\) is said to be \emph{invertible}
if it admits a two-sided inverse.
We define the type of invertibility proofs of \(f\) to be the type
\[\inverse(f) \coloneq \sum_{g : B \to A}(f \circ g = \id) \times (g \circ f = \id)\]
and denote the type of all invertible maps between two types \(A\) and \(B\) by
\(\inv(A, B) \coloneq \sum_{f : A \to B}\inverse(f)\).

A function \(f\) is said to be an \emph{equivalence} if it admits both a left and a
right sided inverse.
We define the type of equivalence proofs of \(f\) to be the type
\[\isequiv(f) \coloneq \sum_{g : B \to A}(f \circ g = \id) \times
  \sum_{h : B \to A}(h \circ f = \id)\]
and similarly denote the type of all equivalences between two types \(A\) and \(B\)
by~\(A \simeq B\).


We recall the following two important facts:
\begin{proposition}
  \label{is-prop-equiv}
  For all functions \(f\), the type of equivalence proofs of \(f\) is a mere proposition.
\end{proposition}

\begin{proposition}
  \label{inv-implies-equiv}
  For all functions \(f\), the types of invertibility and equivalence proofs of f are
  logically equivalent. In other words, a function is invertible if and only if it is
  an equivalence.
\end{proposition}

\begin{lemma}
  \label{inv-id}
  Suppose \(A\) is a type. The type of invertibility proofs of \(\id_{A}\) is equal
  to \(\id_{A} = \id_{A}\).
\end{lemma}

\begin{proof}
  By reassociating the dependent sum type, we see that the type of invertibility proofs of
  \(\id_{A}\) is equal to
  \[\sum \Bigl(G : \hspace{-6pt} \sum_{g : A \to A} \left(g = \id_{A}\right)\Bigr)
    \left( \pr_{1}G = \id_{A}\right).\]
  Since the type \(\sum_{g : A \to A}\left(g = \id_{A}\right)\) is contractible onto
  \(\left(\id_{A}, \refl\right)\), we see the whole type is equal to
  \(\pr_{1}\left(\id_{A}, \refl\right) = \id_{A}\), which is itself equal to
  \(\id_{A} = \id_{A}\).
\end{proof}

\pagebreak

\begin{proposition}
  \label{inv-is-looped-equiv}
  The type \(\inv(A, B)\) is equal to the type \(\sum_{e : A = B} e = e\)
  for all types \(A\) and \(B\).
\end{proposition}

\begin{proof}
  We view \(\inverse\) and \(\isequiv\) as two families of types over \(A \to B\).
  Proposition \ref{is-prop-equiv} then tells us that \(\isequiv\) is a family
  of mere propositions over \(A \to B\), whereas proposition \ref{inv-implies-equiv} in
  particular tells us that we have a family of functions \(\inverse(f) \to \isequiv(f)\) for
  all functions \(f : A \to B\). By proposition~\ref{family-over-subtype}, it then follows
  that the type \(\inv(A, B)\) is equal to the type
  \(\sum_{e : A \simeq B} \inverse(\pr_{1}e)\).

  On the other hand, we first observe that the type \(\sum_{e : A = B} e = e\) is equal to the
  type \(\sum_{e : A \simeq B} e = e\) by univalence. Now, since \(\isequiv\)
  is a family of mere propositions, \(A \simeq B\) is
  a subtype of \(A \to B\). By the subtype identity principle, it follows that the type
  \(e = e\) is equal to the type \(\pr_{1}e = \pr_{1}e\) for all equivalences
  \(e : A \simeq B\).

  To construct the desired equality, it thus suffices to construct an equality between the
  types \(\sum_{e : A \simeq B} \inverse(\pr_{1}e)\) and
  \(\sum_{e : A \simeq B} \pr_{1}e = \pr_{1}e\). By the fiberwise
  equivalence construction, it further suffices to construct an equality between
  \(\inverse(\pr_{1}e)\) and \(\pr_{1}e = \pr_{1}e\) for all equivalences \(e : A \simeq B\).
  The result then follows by equivalence induction and Lemma \ref{inv-id}.
\end{proof}

\begin{theorem}
  The type \(\sum_{A, B : \U}\inv(A, B)\) is equal to the type \(\sphere \to \U\).
\end{theorem}

\begin{proof}
  We first quantify the equality in Proposition
  \ref{inv-is-looped-equiv} over \(B : \U\), obtaining an equality between
  \(\sum_{B : \U}\inv(A, B)\) and \(\sum_{B : \U}\sum_{e : A = B}e = e\).
  Since the type \(\sum_{B : \U}A = B\) is contractible onto
  \(\left(A, \refl_{A}\right)\), the second type is equal to
  \(\refl_{A} = \refl_{A}\). Now quantifying
  over \(A : \U\), we obtain an equality between \(\sum_{A, B : \U}\inv(A, B)\) and
  \(\sum_{A : \U}\refl_{A} = \refl_{A}\), which is equal to \(\sphere \to \U\) by the
  universal property of the sphere.
\end{proof}

\section{\(n\)-invertible functions}

\begin{definition}
  We define a notion of \(n\)-invertibility on \(A \to B\) by induction on~\(n\).
  Let \(f : A \to B\) be a function. We say that \(f\) is
  \(0\)-invertible if there exists a function \(g : B \to A\).
  We say that \(f\) is \((n + 1)\)-invertible if there exist functions \(g : B \to A\) and
  \[r : \prod_{x : A}\prod_{y : B}f x = y \to x = g y,\]
  such that \(r(x, y)\) is \(n\)-invertible for all \(x\) and \(y\).

  \[\ninverse{n + 1}(f) =
    \sum_{g : B \to A}\sum \Bigl(r : \prod_{x : A}\prod_{y : B}f x = y \to x = g y\Bigr)
    \prod_{x : A}\prod_{y : B}\ninverse{n}(r(x, y)).\]
\end{definition}

\begin{lemma}
  The type \(\ninverse{n + 1}(\id_{A})\) is equal to
  \[\prod_{x, y : A}\ninverse{n}(\id_{x = y})\]
  for all \(n : \N\).
\end{lemma}

\begin{proposition}
  The type \(\ninverse{n + 1}(\id_{A})\) is equal to \(\prod_{x : A}\lspace{n + 1}(A, x)\)
  for all~\(n : \N\).
\end{proposition}

\begin{proof}
  We prove the result by induction on \(n\). We make use of the previous lemma in both cases.
  For the base case, we have
  \begin{align*}
    \ninverse{1}(\id_{A}) &= \prod_{x, y : A}\ninverse{0}(\id_{x = y}) =
                              \prod_{x, y : A} (x = y) \to (x = y) = \\
                            &= \prod_{x : A} (x = x) = \prod_{x : A} \lspace{1}(A, x).
  \end{align*}
  For the inductive step, we have
  \begin{align*}
    \ninverse{n + 2}{(\id_{A})} &= \prod_{x, y : A}\ninverse{n + 1}(\id_{x = y}) =
                                  \prod_{x, y : A}\prod_{p : x = y}\lspace{n + 1}(x = y, p) \\
                                &= \prod_{x : A} \lspace{n + 1}(x = x, \refl_{x}) =
                                  \prod_{x : A}\lspace{n + 2}(A, x).
  \end{align*}
\end{proof}

\begin{proposition}
  Let \(n : \N\) and suppose \(f : A \to B\) is \((n + 1)\)-invertible.
  Then \(f\) is \(n\)-invertible.
\end{proposition}

\begin{proof}
  We again prove the result by induction on \(n\). For the base case, suppose \(f : A \to B\)
  is \(1\)-invertible. We then have a function \(g : B \to A\) together with some data,
  but \(g\) itself is enough to show that \(f\) is \(0\)-invertible.

  For the inductive step,
  suppose \(f : A \to B\) is \((n + 2)\)-invertible. We then have a function
  \(g : B \to A\) and a certain dependent function \(r\), such that \(r(x, y)\) is
  \(n + 1\)-invertible for all \(x, y : A\). By the inductive hypothesis, \(r(x, y)\) is
  \(n\)-invertible for all \(x, y : A\), showing that \(f\) is \((n + 1)\)-invertible.
\end{proof}

\begin{corollary}
  Let \(n : \N\) and suppose \(f : A \to B\) is \((n + 1)\)-invertible. Then \(f\) is an
  equivalence.
\end{corollary}

\begin{proof}
  Using the previous proposition, we can show that every \((n + 1)\)-invertible function is
  \(1\)-invertible. Since \(1\)-invertibility coincides with invertibility, this shows that
  it is also an equivalence.
\end{proof}
\end{document}

%%% Local Variables:
%%% mode: LaTeX
%%% TeX-master: t
%%% End:
