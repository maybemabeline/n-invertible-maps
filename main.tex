\documentclass{article}

\usepackage{mathtools}
\usepackage{amsfonts}
\usepackage{amsthm}

\theoremstyle{plain}
\newtheorem{theorem}{Theorem}
\newtheorem{proposition}[theorem]{Proposition}
\newtheorem{lemma}[theorem]{Lemma}
\newtheorem{corollary}[theorem]{Corollary}

\theoremstyle{definition}
\newtheorem{definition}[theorem]{Definition}

\theoremstyle{remark}
\newtheorem{remark}[theorem]{Remark}

\newcommand{\id}{\mathrm{id}}
\newcommand{\Id}{\mathrm{Id}}
\newcommand{\isequiv}{\mathrm{isequiv}}
\newcommand{\pr}{\mathrm{pr}}
\newcommand{\refl}{\mathrm{refl}}
\newcommand{\nsphere}[1]{\mathbb{S}^{#1}}
\newcommand{\sphere}{\nsphere{2}}
\newcommand{\U}{\mathcal{U}}
\newcommand{\N}{\mathbb{N}}
\newcommand{\lspace}[1]{\Omega^{#1}}

\DeclareMathOperator{\inv}{Inv}
\DeclareMathOperator{\inverse}{inverse}
\DeclareMathOperator{\adj}{Adj}
\DeclareMathOperator{\contr}{isContr}

\newcommand{\ninverse}[1]{\inverse_{#1}}

\begin{document}


\section{Basic construction}

\begin{proposition}
  \label{universal-property-sphere}
  For all types \(A\), the type \(\sphere \to A\) is equal to
  \[\sum_{x : A} \refl_{x} = \refl_{x}.\]
\end{proposition}
\begin{proposition}
  \label{family-over-subtype}
  Suppose that \(A\) is a type, \(B\) a family of types over \(A\) and \(P\) a family of mere
  propositions over \(A\), such that we have a family of functions \(B(x) \to P(x)\) for all
  \(x : A\). We then have an equality of types
  \[\sum_{x : A}B(x) = \sum \Bigl(t : \sum_{x : A}P(x)\Bigr) B(\pr_{1}t).\]
\end{proposition}

A function \(f\) is said to be \emph{invertible}
if it admits a two-sided inverse.
We define the type of invertibility proofs of \(f\) to be the type
\[\inverse(f) \coloneq \sum_{g : B \to A}(f \circ g = \id) \times (g \circ f = \id)\]
and denote the type of all invertible maps between two types \(A\) and \(B\) by
\(\inv(A, B) \coloneq \sum_{f : A \to B}\inverse(f)\).

A function \(f\) is said to be an \emph{equivalence} if it admits both a left and a
right sided inverse.
We define the type of equivalence proofs of \(f\) to be the type
\[\isequiv(f) \coloneq \sum_{g : B \to A}(f \circ g = \id) \times
  \sum_{h : B \to A}(h \circ f = \id)\]
and similarly denote the type of all equivalences between two types \(A\) and \(B\)
by~\(A \simeq B\).


We recall the following two important facts:
\begin{proposition}
  \label{is-prop-equiv}
  For all functions \(f\), the type of equivalence proofs of \(f\) is a mere proposition.
\end{proposition}

\begin{proposition}
  \label{inv-implies-equiv}
  For all functions \(f\), the types of invertibility and equivalence proofs of f are
  logically equivalent. In other words, a function is invertible if and only if it is
  an equivalence.
\end{proposition}

\begin{lemma}
  \label{inv-id}
  Suppose \(A\) is a type. The type of invertibility proofs of \(\id_{A}\) is equal
  to \(\id_{A} = \id_{A}\).
\end{lemma}

\begin{proof}
  By reassociating the dependent sum type, we see that the type of invertibility proofs of
  \(\id_{A}\) is equal to
  \[\sum \Bigl(G : \hspace{-6pt} \sum_{g : A \to A} \left(g = \id_{A}\right)\Bigr)
    \left( \pr_{1}G = \id_{A}\right).\]
  Since the type \(\sum_{g : A \to A}\left(g = \id_{A}\right)\) is contractible onto
  \(\left(\id_{A}, \refl\right)\), we see the whole type is equal to
  \(\pr_{1}\left(\id_{A}, \refl\right) = \id_{A}\), which is itself equal to
  \(\id_{A} = \id_{A}\).
\end{proof}

\pagebreak

\begin{proposition}
  \label{inv-is-looped-equiv}
  The type \(\inv(A, B)\) is equal to the type \(\sum_{e : A = B} e = e\)
  for all types \(A\) and \(B\).
\end{proposition}

\begin{proof}
  We view \(\inverse\) and \(\isequiv\) as two families of types over \(A \to B\).
  Proposition \ref{is-prop-equiv} then tells us that \(\isequiv\) is a family
  of mere propositions over \(A \to B\), whereas proposition \ref{inv-implies-equiv} in
  particular tells us that we have a family of functions \(\inverse(f) \to \isequiv(f)\) for
  all functions \(f : A \to B\). By proposition~\ref{family-over-subtype}, it then follows
  that the type \(\inv(A, B)\) is equal to the type
  \(\sum_{e : A \simeq B} \inverse(\pr_{1}e)\).

  On the other hand, we first observe that the type \(\sum_{e : A = B} e = e\) is equal to the
  type \(\sum_{e : A \simeq B} e = e\) by univalence. Now, since \(\isequiv\)
  is a family of mere propositions, \(A \simeq B\) is
  a subtype of \(A \to B\). By the subtype identity principle, it follows that the type
  \(e = e\) is equal to the type \(\pr_{1}e = \pr_{1}e\) for all equivalences
  \(e : A \simeq B\).

  To construct the desired equality, it thus suffices to construct an equality between the
  types \(\sum_{e : A \simeq B} \inverse(\pr_{1}e)\) and
  \(\sum_{e : A \simeq B} \pr_{1}e = \pr_{1}e\). By the fiberwise
  equivalence construction, it further suffices to construct an equality between
  \(\inverse(\pr_{1}e)\) and \(\pr_{1}e = \pr_{1}e\) for all equivalences \(e : A \simeq B\).
  The result then follows by equivalence induction and Lemma \ref{inv-id}.
\end{proof}

\begin{theorem}
  The type \(\sum_{A, B : \U}\inv(A, B)\) is equal to the type \(\sphere \to \U\).
\end{theorem}

\begin{proof}
  We first quantify the equality in Proposition
  \ref{inv-is-looped-equiv} over \(B : \U\), obtaining an equality between
  \(\sum_{B : \U}\inv(A, B)\) and \(\sum_{B : \U}\sum_{e : A = B}e = e\).
  Since the type \(\sum_{B : \U}A = B\) is contractible onto
  \(\left(A, \refl_{A}\right)\), the second type is equal to
  \(\refl_{A} = \refl_{A}\). Now quantifying
  over \(A : \U\), we obtain an equality between \(\sum_{A, B : \U}\inv(A, B)\) and
  \(\sum_{A : \U}\refl_{A} = \refl_{A}\), which is equal to \(\sphere \to \U\) by the
  universal property of the sphere.
\end{proof}

TODO: should this section be cut or be polished up to serve as motivation, introduction?

\section{\(n\)-invertible maps}

\begin{definition}
  We define a notion of \(n\)-invertibility on \(A \to B\) by induction on~\(n\).
  Let \(f : A \to B\) be a function. We say that \(f\) is
  \(0\)-invertible if there exists a function \(g : B \to A\).
  We say that \(f\) is \((n + 1)\)-invertible if there exist functions \(g : B \to A\) and
  \[r : \prod_{x : A}\prod_{y : B}(f(x) = y) \to (x = g(y)),\]
  such that \(r(x, y)\) is \(n\)-invertible for all \(x\) and \(y\).

  We thus define the type of \((n + 1)\)-inverses to be
  \[\ninverse{n + 1}(f) =
    \sum_{g : B \to A}\sum_{r : \adj(f, g)}
    \prod_{x : A}\prod_{y : B}\ninverse{n}(r(x, y)),\]
  where \(\adj(f, g) = \prod_{x : A}\prod_{y : B}(f(x) = y) \to (x = g(y))\).
\end{definition}

\begin{remark}
  We note that \(1\)-invertibility coincides with the ordinary notion of invertibility.
  By path induction, the data of a \(1\)-invertible function \(f : A \to B\)
  consists exactly of a function \(g : B \to A\), together with homotopies of type
  \(\prod_{x : A} x = g(f(x))\) and \(\prod_{y : B} f(g(y)) = y\), witnessing
  that \(f\) is invertible.
\end{remark}

\begin{lemma}
  The type \(\ninverse{n + 1}(\id_{A})\) is equal to
  \[\prod_{x, y : A}\ninverse{n}(\id_{x = y})\]
  for all \(n : \N\).
\end{lemma}

\begin{proof}
  TODO
\end{proof}

\begin{proposition}
  \label{ninv-id}
  The type \(\ninverse{n}(\id_{A})\) is equal to \(\prod_{x : A}\lspace{n}(A, x)\)
  for all~\(n : \N\).
\end{proposition}

\begin{proof}
  We prove the result by induction on \(n\).
  For the base case, we have
  \[\ninverse{0}(\id_{A}) = (A \to A) =
    \prod_{x : A}(A, x) = \prod_{x : A}\lspace{0}(A, x).\]
  For the inductive step, we have
  \begin{align*}
    \ninverse{n + 1}{(\id_{A})} &= \prod_{x, y : A}\ninverse{n}(\id_{x = y}) =
                                  \prod_{x, y : A}\prod_{p : x = y}\lspace{n}(x = y, p) \\
                                &= \prod_{x : A} \lspace{n}(x = x, \refl_{x}) =
                                  \prod_{x : A}\lspace{n + 1}(A, x).
  \end{align*}
\end{proof}

\begin{proposition}
  \label{inverse-system}
  Let \(n : \N\) and suppose \(f : A \to B\) is \((n + 1)\)-invertible.
  Then \(f\) is \(n\)-invertible.
\end{proposition}

\begin{proof}
  We again prove the result by induction on \(n\). For the base case, suppose \(f : A \to B\)
  is \(1\)-invertible. We then have a function \(g : B \to A\) together with some data,
  but \(g\) itself is enough to show that \(f\) is \(0\)-invertible.

  For the inductive step,
  suppose \(f : A \to B\) is \((n + 2)\)-invertible. We then have a function
  \(g : B \to A\) and a certain dependent function \(r\), such that \(r(x, y)\) is
  \((n + 1)\)-invertible for all \(x, y : A\). By the inductive hypothesis, \(r(x, y)\) is
  \(n\)-invertible for all \(x, y : A\), showing that \(f\) is \((n + 1)\)-invertible.
\end{proof}

\begin{corollary}
  Let \(n : \N\) and suppose \(f : A \to B\) is \((n + 1)\)-invertible. Then \(f\) is an
  equivalence.
\end{corollary}

\begin{proof}
  Using the previous proposition, we can show that every \((n + 1)\)-invertible function is
  \(1\)-invertible. Since \(1\)-invertibility coincides with invertibility, this shows that
  it is also an equivalence.
\end{proof}

\begin{theorem}
  \label{ninv-maps-with-domain}
  Let \(\U\) be a universe and \(A : \U\) a type. The type
  \[\sum_{B : \U}\sum_{f : A \to B}\inverse_{n + 1}(f)\] of all \((n + 1)\)-invertible maps
  with domain \(A\) is equal to
  \(\lspace{n + 2}(\U, A)\).
\end{theorem}

\begin{proof}
  Let \(B : \U\) first be a type. Since \((n + 1)\)-invertibility implies equivalence and
  equivalence is a mere proposition, we have
  \[\sum_{f : A \to B}\ninverse{n + 1}(f) = \sum_{f : A \simeq B}\ninverse{n + 1}(\pr_{1}f).\]
  Then, since the type \(\sum_{B : \U}{A \simeq B}\) is contractible, it follows that
  \[\sum_{B : \U}\sum_{f : A \simeq B}\ninverse{n + 1}(\pr_{1}f) =
    \ninverse{n + 1}{(\id_{A})}.\]
  By proposition \ref{ninv-id}, the second type is equal to
  \(\prod_{x : A}\lspace{n + 1}(A, x)\). Finally, since dependent products commute with loop
  spaces, we have
  \[\prod_{x : A}\lspace{n + 1}(A, x) = \lspace{n + 2}(\U, A).\]
  TODO : why exactly is the final claim true?
\end{proof}

\begin{corollary}
  Let \(\U\) be a universe. The type
  \(\sum_{A, B : \U}\sum_{f : A \to B}\ninverse{n + 1}(f)\) is equal to
  \(\nsphere{n + 2} \to \U\).
\end{corollary}

\begin{proof}
  Obtained by quantifying the equality in Theorem \ref{ninv-maps-with-domain} over \(A\)
  and noting
  that \(\sum_{A : \U}\lspace{n + 2}(\U, A) = \nsphere{n + 2} \to \U\) by the universal
  property of the sphere.
\end{proof}

\section{\(\infty\)-invertible maps}

Consider the diagram \(\phi_{n} : \ninverse{n + 1}(f) \to \ninverse{n}(f)\) defined
in Proposition \ref{inverse-system}. We may explicitly define the \(\phi_{n}\) to be
\begin{align*}
  \phi_{0}(g, r, H)     &= g \\
  \phi_{n + 1}(g, r, H) &= (g, r, \lambda x.\lambda y.\phi_{n}(H(x, y))).
\end{align*}
Taking the diagram's limit, we obtain a type \(\ninverse{\infty}(f)\), together with data
\begin{align*}
  \psi_{n}   &: \ninverse{\infty}(f) \to \ninverse{n}(f) \\
  \alpha_{n} &: \phi_{n + 1} \circ \psi_{n + 1} = \psi_{n}.
\end{align*}
A function \(f\), equipped with data of type \(\ninverse{\infty}(f)\), is said to be
\emph{\(\infty\)-invertible}.
Note that every \(\infty\)-invertible function is an equivalence, since it is in particular
\(1\)-invertible by the data of the limit.

\begin{proposition}
  The type \(\ninverse{\infty}(f)\) is a mere proposition for all functions \(f : A \to B\).
\end{proposition}

\begin{proof}
  We show that assuming \(f\) is \(\infty\)-invertible, the type \(\ninverse{\infty}(f)\) is
  contractible. That is to say, we wish to show
  \[\prod_{A, B : \U}\prod_{f : A \to B}\ninverse{\infty}(f) \to
    \contr(\ninverse{\infty}(f)).\]
  Now, since \(\infty\)-invertibility implies equivalence, it suffices to show
  \[\prod_{A, B : \U}\prod_{f : A \to B}\isequiv(f)
    \to \contr(\ninverse{\infty}(\pr_{1}f)).\]
  By equivalence induction, it then suffices to show only
  \[\prod_{A : \U}\contr(\ninverse{\infty}(\id_{A})).\]
  TODO: reword the above
  
  By definition, \(\ninverse{\infty}(\id_{A})\) is the limit of the diagram
  \[\phi_{n} : \ninverse{n + 1}(\id_{A}) \to \ninverse{n}(\id_{A}).\]
  TODO: we could now try to use Proposition \ref{ninv-id} somehow, this seems related to
  the contractibility of \(\nsphere{\infty}\).
\end{proof}

\begin{proposition}
  The type of \(\infty\)-invertibility proofs satisfies the recursive equation
  \[\ninverse{\infty}(f) =
    \sum_{g : B \to A}\sum_{r : \adj(f, g)}
    \prod_{x : A}\prod_{y : B}\ninverse{\infty}(r(x, y))\]
  for all \(f : A \to B\).
\end{proposition}

\begin{proof}
  We obtain the desired equality by showing the second type satisfies the universal property
  of the limit. Suppose then that \(P\) is a type, equipped with a cocone
  \begin{align*}
    \rho_{n}     &: P \to \ninverse{n}(f) \\
    \beta_{n} &: \phi_{n + 1} \circ \rho_{n + 1} = \rho_{n}.
  \end{align*}
  TODO: not sure if this is a right approach.
\end{proof}


\end{document}

%%% Local Variables:
%%% mode: LaTeX
%%% TeX-master: t
%%% End:
